% Options for packages loaded elsewhere
% Options for packages loaded elsewhere
\PassOptionsToPackage{unicode}{hyperref}
\PassOptionsToPackage{hyphens}{url}
\PassOptionsToPackage{dvipsnames,svgnames,x11names}{xcolor}
%
\documentclass[
  letterpaper,
  DIV=11,
  numbers=noendperiod]{scrartcl}
\usepackage{xcolor}
\usepackage[margin=1in]{geometry}
\usepackage{amsmath,amssymb}
\setcounter{secnumdepth}{5}
\usepackage{iftex}
\ifPDFTeX
  \usepackage[T1]{fontenc}
  \usepackage[utf8]{inputenc}
  \usepackage{textcomp} % provide euro and other symbols
\else % if luatex or xetex
  \usepackage{unicode-math} % this also loads fontspec
  \defaultfontfeatures{Scale=MatchLowercase}
  \defaultfontfeatures[\rmfamily]{Ligatures=TeX,Scale=1}
\fi
\usepackage{lmodern}
\ifPDFTeX\else
  % xetex/luatex font selection
  \setmainfont[]{Latin Modern Roman}
  \setsansfont[]{Latin Modern Sans}
  \setmonofont[]{Latin Modern Mono}
\fi
% Use upquote if available, for straight quotes in verbatim environments
\IfFileExists{upquote.sty}{\usepackage{upquote}}{}
\IfFileExists{microtype.sty}{% use microtype if available
  \usepackage[]{microtype}
  \UseMicrotypeSet[protrusion]{basicmath} % disable protrusion for tt fonts
}{}
\makeatletter
\@ifundefined{KOMAClassName}{% if non-KOMA class
  \IfFileExists{parskip.sty}{%
    \usepackage{parskip}
  }{% else
    \setlength{\parindent}{0pt}
    \setlength{\parskip}{6pt plus 2pt minus 1pt}}
}{% if KOMA class
  \KOMAoptions{parskip=half}}
\makeatother
% Make \paragraph and \subparagraph free-standing
\makeatletter
\ifx\paragraph\undefined\else
  \let\oldparagraph\paragraph
  \renewcommand{\paragraph}{
    \@ifstar
      \xxxParagraphStar
      \xxxParagraphNoStar
  }
  \newcommand{\xxxParagraphStar}[1]{\oldparagraph*{#1}\mbox{}}
  \newcommand{\xxxParagraphNoStar}[1]{\oldparagraph{#1}\mbox{}}
\fi
\ifx\subparagraph\undefined\else
  \let\oldsubparagraph\subparagraph
  \renewcommand{\subparagraph}{
    \@ifstar
      \xxxSubParagraphStar
      \xxxSubParagraphNoStar
  }
  \newcommand{\xxxSubParagraphStar}[1]{\oldsubparagraph*{#1}\mbox{}}
  \newcommand{\xxxSubParagraphNoStar}[1]{\oldsubparagraph{#1}\mbox{}}
\fi
\makeatother

\usepackage{color}
\usepackage{fancyvrb}
\newcommand{\VerbBar}{|}
\newcommand{\VERB}{\Verb[commandchars=\\\{\}]}
\DefineVerbatimEnvironment{Highlighting}{Verbatim}{commandchars=\\\{\}}
% Add ',fontsize=\small' for more characters per line
\usepackage{framed}
\definecolor{shadecolor}{RGB}{241,243,245}
\newenvironment{Shaded}{\begin{snugshade}}{\end{snugshade}}
\newcommand{\AlertTok}[1]{\textcolor[rgb]{0.68,0.00,0.00}{#1}}
\newcommand{\AnnotationTok}[1]{\textcolor[rgb]{0.37,0.37,0.37}{#1}}
\newcommand{\AttributeTok}[1]{\textcolor[rgb]{0.40,0.45,0.13}{#1}}
\newcommand{\BaseNTok}[1]{\textcolor[rgb]{0.68,0.00,0.00}{#1}}
\newcommand{\BuiltInTok}[1]{\textcolor[rgb]{0.00,0.23,0.31}{#1}}
\newcommand{\CharTok}[1]{\textcolor[rgb]{0.13,0.47,0.30}{#1}}
\newcommand{\CommentTok}[1]{\textcolor[rgb]{0.37,0.37,0.37}{#1}}
\newcommand{\CommentVarTok}[1]{\textcolor[rgb]{0.37,0.37,0.37}{\textit{#1}}}
\newcommand{\ConstantTok}[1]{\textcolor[rgb]{0.56,0.35,0.01}{#1}}
\newcommand{\ControlFlowTok}[1]{\textcolor[rgb]{0.00,0.23,0.31}{\textbf{#1}}}
\newcommand{\DataTypeTok}[1]{\textcolor[rgb]{0.68,0.00,0.00}{#1}}
\newcommand{\DecValTok}[1]{\textcolor[rgb]{0.68,0.00,0.00}{#1}}
\newcommand{\DocumentationTok}[1]{\textcolor[rgb]{0.37,0.37,0.37}{\textit{#1}}}
\newcommand{\ErrorTok}[1]{\textcolor[rgb]{0.68,0.00,0.00}{#1}}
\newcommand{\ExtensionTok}[1]{\textcolor[rgb]{0.00,0.23,0.31}{#1}}
\newcommand{\FloatTok}[1]{\textcolor[rgb]{0.68,0.00,0.00}{#1}}
\newcommand{\FunctionTok}[1]{\textcolor[rgb]{0.28,0.35,0.67}{#1}}
\newcommand{\ImportTok}[1]{\textcolor[rgb]{0.00,0.46,0.62}{#1}}
\newcommand{\InformationTok}[1]{\textcolor[rgb]{0.37,0.37,0.37}{#1}}
\newcommand{\KeywordTok}[1]{\textcolor[rgb]{0.00,0.23,0.31}{\textbf{#1}}}
\newcommand{\NormalTok}[1]{\textcolor[rgb]{0.00,0.23,0.31}{#1}}
\newcommand{\OperatorTok}[1]{\textcolor[rgb]{0.37,0.37,0.37}{#1}}
\newcommand{\OtherTok}[1]{\textcolor[rgb]{0.00,0.23,0.31}{#1}}
\newcommand{\PreprocessorTok}[1]{\textcolor[rgb]{0.68,0.00,0.00}{#1}}
\newcommand{\RegionMarkerTok}[1]{\textcolor[rgb]{0.00,0.23,0.31}{#1}}
\newcommand{\SpecialCharTok}[1]{\textcolor[rgb]{0.37,0.37,0.37}{#1}}
\newcommand{\SpecialStringTok}[1]{\textcolor[rgb]{0.13,0.47,0.30}{#1}}
\newcommand{\StringTok}[1]{\textcolor[rgb]{0.13,0.47,0.30}{#1}}
\newcommand{\VariableTok}[1]{\textcolor[rgb]{0.07,0.07,0.07}{#1}}
\newcommand{\VerbatimStringTok}[1]{\textcolor[rgb]{0.13,0.47,0.30}{#1}}
\newcommand{\WarningTok}[1]{\textcolor[rgb]{0.37,0.37,0.37}{\textit{#1}}}

\usepackage{longtable,booktabs,array}
\usepackage{calc} % for calculating minipage widths
% Correct order of tables after \paragraph or \subparagraph
\usepackage{etoolbox}
\makeatletter
\patchcmd\longtable{\par}{\if@noskipsec\mbox{}\fi\par}{}{}
\makeatother
% Allow footnotes in longtable head/foot
\IfFileExists{footnotehyper.sty}{\usepackage{footnotehyper}}{\usepackage{footnote}}
\makesavenoteenv{longtable}
\usepackage{graphicx}
\makeatletter
\newsavebox\pandoc@box
\newcommand*\pandocbounded[1]{% scales image to fit in text height/width
  \sbox\pandoc@box{#1}%
  \Gscale@div\@tempa{\textheight}{\dimexpr\ht\pandoc@box+\dp\pandoc@box\relax}%
  \Gscale@div\@tempb{\linewidth}{\wd\pandoc@box}%
  \ifdim\@tempb\p@<\@tempa\p@\let\@tempa\@tempb\fi% select the smaller of both
  \ifdim\@tempa\p@<\p@\scalebox{\@tempa}{\usebox\pandoc@box}%
  \else\usebox{\pandoc@box}%
  \fi%
}
% Set default figure placement to htbp
\def\fps@figure{htbp}
\makeatother


% definitions for citeproc citations
\NewDocumentCommand\citeproctext{}{}
\NewDocumentCommand\citeproc{mm}{%
  \begingroup\def\citeproctext{#2}\cite{#1}\endgroup}
\makeatletter
 % allow citations to break across lines
 \let\@cite@ofmt\@firstofone
 % avoid brackets around text for \cite:
 \def\@biblabel#1{}
 \def\@cite#1#2{{#1\if@tempswa , #2\fi}}
\makeatother
\newlength{\cslhangindent}
\setlength{\cslhangindent}{1.5em}
\newlength{\csllabelwidth}
\setlength{\csllabelwidth}{3em}
\newenvironment{CSLReferences}[2] % #1 hanging-indent, #2 entry-spacing
 {\begin{list}{}{%
  \setlength{\itemindent}{0pt}
  \setlength{\leftmargin}{0pt}
  \setlength{\parsep}{0pt}
  % turn on hanging indent if param 1 is 1
  \ifodd #1
   \setlength{\leftmargin}{\cslhangindent}
   \setlength{\itemindent}{-1\cslhangindent}
  \fi
  % set entry spacing
  \setlength{\itemsep}{#2\baselineskip}}}
 {\end{list}}
\usepackage{calc}
\newcommand{\CSLBlock}[1]{\hfill\break\parbox[t]{\linewidth}{\strut\ignorespaces#1\strut}}
\newcommand{\CSLLeftMargin}[1]{\parbox[t]{\csllabelwidth}{\strut#1\strut}}
\newcommand{\CSLRightInline}[1]{\parbox[t]{\linewidth - \csllabelwidth}{\strut#1\strut}}
\newcommand{\CSLIndent}[1]{\hspace{\cslhangindent}#1}



\setlength{\emergencystretch}{3em} % prevent overfull lines

\providecommand{\tightlist}{%
  \setlength{\itemsep}{0pt}\setlength{\parskip}{0pt}}



 


% Colors and section/title styling using KOMA-Script interfaces
\usepackage{xcolor}
\definecolor{sectionblue}{HTML}{2563eb}

% KOMA: headings and title/subtitle colors
\setkomafont{title}{\color{sectionblue}\bfseries\Huge}
\setkomafont{subtitle}{\color{sectionblue}\large}
\setkomafont{section}{\color{sectionblue}\bfseries\Large}
\setkomafont{subsection}{\color{sectionblue}\bfseries\large}

% Code block styling via Shaded redefinition
\usepackage{tcolorbox}
\tcbuselibrary{skins,breakable}
\definecolor{codebg}{HTML}{F0F8FF}
\renewenvironment{Shaded}{%
  \begin{tcolorbox}[%
    enhanced,%
    colback=codebg,%
    colframe=codebg,%
    borderline west={3pt}{0pt}{sectionblue},%
    fontupper=\small\ttfamily,% reduce font size and force monospace for code
    boxrule=0pt,%
    arc=0pt,%
    boxsep=5pt,%
    left=2mm,%
    right=2mm,%
    top=2mm,%
    bottom=2mm% 
  ]% 
}{%
  \end{tcolorbox}%
}
\KOMAoption{captions}{tableheading}
\makeatletter
\@ifpackageloaded{caption}{}{\usepackage{caption}}
\AtBeginDocument{%
\ifdefined\contentsname
  \renewcommand*\contentsname{Table of contents}
\else
  \newcommand\contentsname{Table of contents}
\fi
\ifdefined\listfigurename
  \renewcommand*\listfigurename{List of Figures}
\else
  \newcommand\listfigurename{List of Figures}
\fi
\ifdefined\listtablename
  \renewcommand*\listtablename{List of Tables}
\else
  \newcommand\listtablename{List of Tables}
\fi
\ifdefined\figurename
  \renewcommand*\figurename{Figure}
\else
  \newcommand\figurename{Figure}
\fi
\ifdefined\tablename
  \renewcommand*\tablename{Table}
\else
  \newcommand\tablename{Table}
\fi
}
\@ifpackageloaded{float}{}{\usepackage{float}}
\floatstyle{ruled}
\@ifundefined{c@chapter}{\newfloat{codelisting}{h}{lop}}{\newfloat{codelisting}{h}{lop}[chapter]}
\floatname{codelisting}{Listing}
\newcommand*\listoflistings{\listof{codelisting}{List of Listings}}
\makeatother
\makeatletter
\makeatother
\makeatletter
\@ifpackageloaded{caption}{}{\usepackage{caption}}
\@ifpackageloaded{subcaption}{}{\usepackage{subcaption}}
\makeatother
\usepackage{bookmark}
\IfFileExists{xurl.sty}{\usepackage{xurl}}{} % add URL line breaks if available
\urlstyle{same}
\hypersetup{
  pdftitle={Geometric Analysis of Sensor Drift},
  colorlinks=true,
  linkcolor={blue},
  filecolor={Maroon},
  citecolor={Blue},
  urlcolor={Blue},
  pdfcreator={LaTeX via pandoc}}


\title{Geometric Analysis of Sensor Drift}
\usepackage{etoolbox}
\makeatletter
\providecommand{\subtitle}[1]{% add subtitle to \maketitle
  \apptocmd{\@title}{\par {\large #1 \par}}{}{}
}
\makeatother
\subtitle{A Principal Component Perspective on Chemical Signature
Stability}
\author{}
\date{}
\begin{document}
\maketitle

\renewcommand*\contentsname{Table of contents}
{
\hypersetup{linkcolor=}
\setcounter{tocdepth}{3}
\tableofcontents
}

\section{Project Introduction and Problem
Description}\label{project-introduction-and-problem-description}

\subsection{Project Overview}\label{project-overview}

This project investigates the phenomenon of sensor drift in gas sensor
arrays through the lens of \textbf{unsupervised learning}, specifically
employing \textbf{Principal Component Analysis (PCA)} as the primary
dimensionality reduction technique. The core challenge addresses how
chemical sensor measurements, which exist in a 128-dimensional space,
change their response patterns over time due to sensor aging and
environmental factors---a critical problem in chemical detection systems
that affects reliability and accuracy.

The project reframes the traditional calibration problem as a geometric
analysis task: understanding how the low-dimensional manifold occupied
by chemical signatures transforms over time in high-dimensional
measurement space. By applying PCA and related techniques, we aim to
identify stable subspaces that remain invariant despite temporal drift,
ultimately developing methods for drift correction and improved chemical
classification.

\subsection{Type of Learning and Task}\label{type-of-learning-and-task}

\textbf{Learning Paradigm:} Unsupervised Learning

\begin{itemize}
\tightlist
\item
  No labeled drift patterns are provided; we discover structure from the
  data itself
\item
  Focus on understanding intrinsic data geometry and temporal evolution
\end{itemize}

\textbf{Primary Algorithms:}

\begin{itemize}
\tightlist
\item
  \textbf{Principal Component Analysis (PCA)}: For dimensionality
  reduction and identifying dominant variance directions
\item
  \textbf{Clustering algorithms} (K-means, hierarchical): For grouping
  chemical signatures
\item
  \textbf{Procrustes Analysis}: For geometric alignment and drift
  correction
\end{itemize}

\textbf{Task Type:}

\begin{itemize}
\tightlist
\item
  \textbf{Dimensionality Reduction}: Reducing 128-dimensional sensor
  readings to a manageable subspace
\item
  \textbf{Anomaly Detection}: Identifying drift patterns as deviations
  from expected behavior
\item
  \textbf{Pattern Recognition}: Discovering invariant features across
  temporal batches
\end{itemize}

\subsection{Project Goals and
Motivation}\label{project-goals-and-motivation}

\textbf{Primary Goal:} To develop a mathematical framework for
understanding and correcting sensor drift through principal component
analysis, achieving at least a 30\% reduction in drift-induced
classification errors.

\textbf{Why This Matters:}

\begin{enumerate}
\def\labelenumi{\arabic{enumi}.}
\item
  \textbf{Industrial Relevance}: Gas sensor arrays are widely used in
  environmental monitoring, food quality control, and safety systems.
  Sensor drift causes frequent recalibration needs, increasing
  operational costs.
\item
  \textbf{Scientific Innovation}: By treating drift as a geometric
  transformation in PC space rather than noise, we can develop more
  principled correction methods that preserve chemical signature
  integrity.
\item
  \textbf{Practical Impact}: A successful drift correction method would
  extend sensor array lifetime, reduce maintenance requirements, and
  improve long-term reliability of chemical detection systems.
\end{enumerate}

\textbf{Specific Objectives:}

\begin{itemize}
\tightlist
\item
  Prove that chemical signatures occupy a low-dimensional manifold (5-8
  dimensions) within the 128-dimensional measurement space
\item
  Quantify the stability of different principal components over 36
  months
\item
  Develop a mathematical model of drift as geometric transformations
\item
  Create a Procrustes-based correction algorithm achieving 67\%
  reduction in drift effects
\item
  Validate improvements using multiple clustering quality metrics
\end{itemize}

\subsection{Data Source and Citation}\label{data-source-and-citation}

\textbf{Dataset:} Gas Sensor Array Drift Dataset

\textbf{Source:} UCI Machine Learning Repository

\textbf{Full Citation:} Vergara, A., Vembu, S., Ayhan, T., Ryan, M. A.,
Homer, M. L., \& Huerta, R. (2012). \emph{Gas Sensor Array Drift
Dataset}. UCI Machine Learning Repository.
https://doi.org/10.24432/C5ZS4K

\textbf{Data Description:} The dataset contains measurements from an
array of 128 metal oxide gas sensors exposed to six different gaseous
substances (Ethanol, Ethylene, Ammonia, Acetaldehyde, Acetone, and
Toluene) at various concentrations. Data was collected over 36 months in
five distinct batches (months 1, 5, 10, 15, and 20), capturing the
natural drift phenomenon as sensors age. Each measurement consists of
128 features representing individual sensor responses, with
approximately 13,910 total observations across all batches.

\textbf{Data Collection Method:} Measurements were obtained in a
controlled laboratory environment using a standardized gas delivery
system. Each batch represents a different time point in the sensor
array's lifetime, allowing us to study temporal drift patterns
systematically.

\subsection{Related Work and Novel
Contributions}\label{related-work-and-novel-contributions}

Vergara et al. (2012a) introduced the Gas Sensor Array Drift Dataset and
focused on drift compensation for supervised classification tasks,
employing ensemble methods to maintain gas identification accuracy
despite sensor aging. Their work, along with subsequent studies
(Rodriguez-Lujan et al. (2014)), primarily addressed drift as a
challenge for prediction accuracy rather than investigating the
geometric properties of drift itself.

This project diverges from existing supervised approaches by examining
drift through an unsupervised, geometric lens. Specifically, we
investigate which principal components of the 128-dimensional sensor
space exhibit stability across time batches and characterize drift as
manifold deformations in reduced-dimensional space. By computing angular
distances between principal component vectors across different
measurement campaigns, we develop stability rankings that identify
drift-resistant subspaces---an analysis not present in prior work on
this dataset.

\section{Data Loading \& Inspection}\label{data-loading-inspection}

\subsection{Data Import Challenges}\label{data-import-challenges}

The original dataset from UCI uses LibSVM sparse format, which required
careful interpretation. Initially, we misidentified feature 1 (with
values around 15,000-670,000) as concentration data, when it's actually
ΔR₁ - the absolute resistance change in Ohms for the first sensor. The
UCI documentation clarified that this is the \textbf{original dataset
version without concentration values}. All 128 features represent sensor
measurements: 16 chemical sensors × 8 features each (2 steady-state + 6
transient features). Understanding this structure was crucial for proper
data processing and column naming
(S\{sensor\}\emph{F\{feature\}}\{type\} format).

To handle this complex data conversion systematically, we created a
dedicated processing script
\texttt{src/sensor\_drift/process\_dat\_to\_csv.py} that converts the
raw LibSVM format (.dat files) to a properly structured CSV with
meaningful column names, without applying any data modifications.

\subsection{Data Inspection}\label{data-inspection}

In this section, we'll load the gas sensor drift dataset and perform
some initial inspection of the data to understand its structure, size,
and key characteristics.

\begin{Shaded}
\begin{Highlighting}[]
\ImportTok{import}\NormalTok{ pandas }\ImportTok{as}\NormalTok{ pd}
\ImportTok{import}\NormalTok{ numpy }\ImportTok{as}\NormalTok{ np}
\ImportTok{from}\NormalTok{ pathlib }\ImportTok{import}\NormalTok{ Path}
\ImportTok{import}\NormalTok{ matplotlib.pyplot }\ImportTok{as}\NormalTok{ plt}
\ImportTok{import}\NormalTok{ seaborn }\ImportTok{as}\NormalTok{ sns}

\CommentTok{\# Set style for better visualizations}
\NormalTok{plt.style.use(}\StringTok{\textquotesingle{}seaborn{-}v0\_8{-}darkgrid\textquotesingle{}}\NormalTok{)}
\NormalTok{sns.set\_palette(}\StringTok{"husl"}\NormalTok{)}

\CommentTok{\# Load data}
\NormalTok{data\_path }\OperatorTok{=}\NormalTok{ Path(}\StringTok{"../data/processed/sensor\_data.csv"}\NormalTok{)}
\NormalTok{df }\OperatorTok{=}\NormalTok{ pd.read\_csv(data\_path)}

\CommentTok{\# Basic Dataset Information}
\BuiltInTok{print}\NormalTok{(}\StringTok{"="}\OperatorTok{*}\DecValTok{60}\NormalTok{)}
\BuiltInTok{print}\NormalTok{(}\StringTok{"DATASET OVERVIEW"}\NormalTok{)}
\BuiltInTok{print}\NormalTok{(}\StringTok{"="}\OperatorTok{*}\DecValTok{60}\NormalTok{)}
\BuiltInTok{print}\NormalTok{(}\SpecialStringTok{f"Dataset shape: }\SpecialCharTok{\{}\NormalTok{df}\SpecialCharTok{.}\NormalTok{shape[}\DecValTok{0}\NormalTok{]}\SpecialCharTok{:,\}}\SpecialStringTok{ rows × }\SpecialCharTok{\{}\NormalTok{df}\SpecialCharTok{.}\NormalTok{shape[}\DecValTok{1}\NormalTok{]}\SpecialCharTok{\}}\SpecialStringTok{ columns"}\NormalTok{)}
\BuiltInTok{print}\NormalTok{(}\SpecialStringTok{f"}\CharTok{\textbackslash{}n}\SpecialStringTok{Memory usage: }\SpecialCharTok{\{}\NormalTok{df}\SpecialCharTok{.}\NormalTok{memory\_usage(deep}\OperatorTok{=}\VariableTok{True}\NormalTok{)}\SpecialCharTok{.}\BuiltInTok{sum}\NormalTok{() }\OperatorTok{/} \DecValTok{1024}\OperatorTok{**}\DecValTok{2}\SpecialCharTok{:.2f\}}\SpecialStringTok{ MB"}\NormalTok{)}

\CommentTok{\# Data types distribution}
\BuiltInTok{print}\NormalTok{(}\SpecialStringTok{f"}\CharTok{\textbackslash{}n}\SpecialStringTok{Data types distribution:"}\NormalTok{)}
\ControlFlowTok{for}\NormalTok{ dtype, count }\KeywordTok{in}\NormalTok{ df.dtypes.value\_counts().items():}
    \BuiltInTok{print}\NormalTok{(}\SpecialStringTok{f"  }\SpecialCharTok{\{}\NormalTok{dtype}\SpecialCharTok{\}}\SpecialStringTok{: }\SpecialCharTok{\{}\NormalTok{count}\SpecialCharTok{\}}\SpecialStringTok{ columns"}\NormalTok{)}

\CommentTok{\# Column information}
\BuiltInTok{print}\NormalTok{(}\StringTok{"}\CharTok{\textbackslash{}n}\StringTok{Column categories:"}\NormalTok{)}
\NormalTok{sensor\_cols }\OperatorTok{=}\NormalTok{ [col }\ControlFlowTok{for}\NormalTok{ col }\KeywordTok{in}\NormalTok{ df.columns }\ControlFlowTok{if}\NormalTok{ col.startswith(}\StringTok{\textquotesingle{}S\textquotesingle{}}\NormalTok{)]}
\NormalTok{metadata\_cols }\OperatorTok{=}\NormalTok{ [col }\ControlFlowTok{for}\NormalTok{ col }\KeywordTok{in}\NormalTok{ df.columns }\ControlFlowTok{if}\NormalTok{ col }\KeywordTok{not} \KeywordTok{in}\NormalTok{ sensor\_cols]}
\BuiltInTok{print}\NormalTok{(}\SpecialStringTok{f"  Sensor features: }\SpecialCharTok{\{}\BuiltInTok{len}\NormalTok{(sensor\_cols)}\SpecialCharTok{\}}\SpecialStringTok{ columns"}\NormalTok{)}
\BuiltInTok{print}\NormalTok{(}\SpecialStringTok{f"  Metadata columns: }\SpecialCharTok{\{}\BuiltInTok{len}\NormalTok{(metadata\_cols)}\SpecialCharTok{\}}\SpecialStringTok{ columns"}\NormalTok{)}
\BuiltInTok{print}\NormalTok{(}\SpecialStringTok{f"  Metadata column names: }\SpecialCharTok{\{}\NormalTok{metadata\_cols}\SpecialCharTok{\}}\SpecialStringTok{"}\NormalTok{)}

\CommentTok{\# Display sample data}
\BuiltInTok{print}\NormalTok{(}\StringTok{"}\CharTok{\textbackslash{}n}\StringTok{Sample data (first 3 rows, selected columns):"}\NormalTok{)}
\NormalTok{sample\_cols }\OperatorTok{=}\NormalTok{ [}\StringTok{\textquotesingle{}gas\_type\textquotesingle{}}\NormalTok{, }\StringTok{\textquotesingle{}gas\_name\textquotesingle{}}\NormalTok{, }\StringTok{\textquotesingle{}batch\textquotesingle{}}\NormalTok{] }\OperatorTok{+}\NormalTok{ sensor\_cols[:}\DecValTok{3}\NormalTok{]}
\BuiltInTok{print}\NormalTok{(df[sample\_cols].head(}\DecValTok{3}\NormalTok{).to\_string(index}\OperatorTok{=}\VariableTok{False}\NormalTok{))}
\end{Highlighting}
\end{Shaded}

\begin{verbatim}
============================================================
DATASET OVERVIEW
============================================================
Dataset shape: 13,910 rows × 131 columns

Memory usage: 14.66 MB

Data types distribution:
  float64: 128 columns
  int64: 2 columns
  object: 1 columns

Column categories:
  Sensor features: 128 columns
  Metadata columns: 3 columns
  Metadata column names: ['gas_type', 'gas_name', 'batch']

Sample data (first 3 rows, selected columns):
 gas_type gas_name  batch  S01_F1_DR  S01_F2_DR_norm  S01_F3_EMAi_001
        1  Ethanol      1 15596.1621        1.868245         2.371604
        1  Ethanol      1 26402.0704        2.532401         5.411209
        1  Ethanol      1 42103.5820        3.454189         8.198175
\end{verbatim}

Next, we'll check for potential data quality issues:

\begin{itemize}
\tightlist
\item
  Missing values - Identify any null or NaN entries
\item
  Data types - Verify all sensor columns are numeric
\item
  Duplicates - Detect repeated samples that could bias analysis
\item
  Temporal consistency - Ensure batch labels are sequential and complete
\item
  Value ranges - Identify impossible values (e.g., negative sensor
  readings)
\item
  Sample balance - Verify adequate representation across gas types and
  concentrations''
\end{itemize}

\begin{Shaded}
\begin{Highlighting}[]

\CommentTok{\# Store original shape for comparison}
\NormalTok{original\_shape }\OperatorTok{=}\NormalTok{ df.shape}
\BuiltInTok{print}\NormalTok{(}\SpecialStringTok{f"Original dataset: }\SpecialCharTok{\{}\NormalTok{original\_shape[}\DecValTok{0}\NormalTok{]}\SpecialCharTok{:,\}}\SpecialStringTok{ rows × }\SpecialCharTok{\{}\NormalTok{original\_shape[}\DecValTok{1}\NormalTok{]}\SpecialCharTok{\}}\SpecialStringTok{ columns"}\NormalTok{)}

\CommentTok{\# 1. Check for missing values}
\NormalTok{missing\_check }\OperatorTok{=}\NormalTok{ df.isnull().}\BuiltInTok{sum}\NormalTok{()}
\BuiltInTok{print}\NormalTok{(}\SpecialStringTok{f"}\CharTok{\textbackslash{}n}\SpecialStringTok{1. Missing Values: }\SpecialCharTok{\{}\NormalTok{missing\_check}\SpecialCharTok{.}\BuiltInTok{sum}\NormalTok{()}\SpecialCharTok{\}}\SpecialStringTok{ total"}\NormalTok{)}

\CommentTok{\# 2. Data types check}
\NormalTok{non\_numeric\_sensors }\OperatorTok{=}\NormalTok{ [col }\ControlFlowTok{for}\NormalTok{ col }\KeywordTok{in}\NormalTok{ sensor\_cols }
                       \ControlFlowTok{if} \KeywordTok{not}\NormalTok{ pd.api.types.is\_numeric\_dtype(df[col])]}
\BuiltInTok{print}\NormalTok{(}\SpecialStringTok{f"2. Non{-}numeric Sensor Columns: "}\NormalTok{,}
      \SpecialStringTok{f"}\SpecialCharTok{\{}\NormalTok{non\_numeric\_sensors }\ControlFlowTok{if}\NormalTok{ non\_numeric\_sensors }\ControlFlowTok{else} \StringTok{\textquotesingle{}All numeric ✓\textquotesingle{}}\SpecialCharTok{\}}\SpecialStringTok{"}\NormalTok{)}

\CommentTok{\# 3. Check for duplicates}
\NormalTok{duplicates }\OperatorTok{=}\NormalTok{ df.duplicated()}
\BuiltInTok{print}\NormalTok{(}\SpecialStringTok{f"3. Duplicate Rows: }\SpecialCharTok{\{}\NormalTok{duplicates}\SpecialCharTok{.}\BuiltInTok{sum}\NormalTok{()}\SpecialCharTok{\}}\SpecialStringTok{"}\NormalTok{)}

\CommentTok{\# 4. Temporal consistency}
\NormalTok{batch\_values }\OperatorTok{=} \BuiltInTok{sorted}\NormalTok{(df[}\StringTok{\textquotesingle{}batch\textquotesingle{}}\NormalTok{].unique())}
\NormalTok{expected\_batches }\OperatorTok{=} \BuiltInTok{list}\NormalTok{(}\BuiltInTok{range}\NormalTok{(}\DecValTok{1}\NormalTok{, }\BuiltInTok{max}\NormalTok{(batch\_values)}\OperatorTok{+}\DecValTok{1}\NormalTok{))}
\NormalTok{missing\_batches }\OperatorTok{=} \BuiltInTok{set}\NormalTok{(expected\_batches) }\OperatorTok{{-}} \BuiltInTok{set}\NormalTok{(batch\_values)}
\BuiltInTok{print}\NormalTok{(}\SpecialStringTok{f"4. Temporal Consistency:"}\NormalTok{)}
\BuiltInTok{print}\NormalTok{(}\SpecialStringTok{f"   Batches present: }\SpecialCharTok{\{}\BuiltInTok{min}\NormalTok{(batch\_values)}\SpecialCharTok{\}}\SpecialStringTok{{-}}\SpecialCharTok{\{}\BuiltInTok{max}\NormalTok{(batch\_values)}\SpecialCharTok{\}}\SpecialStringTok{"}\NormalTok{)}
\BuiltInTok{print}\NormalTok{(}\SpecialStringTok{f"   Missing batches: }\SpecialCharTok{\{}\BuiltInTok{list}\NormalTok{(missing\_batches) }\ControlFlowTok{if}\NormalTok{ missing\_batches }\ControlFlowTok{else} \StringTok{\textquotesingle{}None ✓\textquotesingle{}}\SpecialCharTok{\}}\SpecialStringTok{"}\NormalTok{)}

\CommentTok{\# 5. Value ranges}
\NormalTok{sensor\_data }\OperatorTok{=}\NormalTok{ df[sensor\_cols]}
\BuiltInTok{print}\NormalTok{(}\SpecialStringTok{f"5. Sensor Value Ranges:"}\NormalTok{)}
\BuiltInTok{print}\NormalTok{(}\SpecialStringTok{f"   Min: }\SpecialCharTok{\{}\NormalTok{sensor\_data}\SpecialCharTok{.}\BuiltInTok{min}\NormalTok{()}\SpecialCharTok{.}\BuiltInTok{min}\NormalTok{()}\SpecialCharTok{:.2f\}}\SpecialStringTok{"}\NormalTok{)}
\BuiltInTok{print}\NormalTok{(}\SpecialStringTok{f"   Max: }\SpecialCharTok{\{}\NormalTok{sensor\_data}\SpecialCharTok{.}\BuiltInTok{max}\NormalTok{()}\SpecialCharTok{.}\BuiltInTok{max}\NormalTok{()}\SpecialCharTok{:.2f\}}\SpecialStringTok{"}\NormalTok{)}
\NormalTok{negative\_sensors }\OperatorTok{=}\NormalTok{ (sensor\_data }\OperatorTok{\textless{}} \DecValTok{0}\NormalTok{).}\BuiltInTok{any}\NormalTok{()}
\BuiltInTok{print}\NormalTok{(}\SpecialStringTok{f"   Sensors with negative values: }\SpecialCharTok{\{}\NormalTok{negative\_sensors}\SpecialCharTok{.}\BuiltInTok{sum}\NormalTok{()}\SpecialCharTok{\}}\SpecialStringTok{"}\NormalTok{)}

\CommentTok{\# 6. Sample balance}
\NormalTok{gas\_counts }\OperatorTok{=}\NormalTok{ df[}\StringTok{\textquotesingle{}gas\_name\textquotesingle{}}\NormalTok{].value\_counts()}
\BuiltInTok{print}\NormalTok{(}\SpecialStringTok{f"6. Sample Balance:"}\NormalTok{)}
\BuiltInTok{print}\NormalTok{(}\SpecialStringTok{f"   Gas types: }\SpecialCharTok{\{}\BuiltInTok{len}\NormalTok{(gas\_counts)}\SpecialCharTok{\}}\SpecialStringTok{ types"}\NormalTok{)}
\BuiltInTok{print}\NormalTok{(}\SpecialStringTok{f"   Min samples: }\SpecialCharTok{\{}\NormalTok{gas\_counts}\SpecialCharTok{.}\BuiltInTok{min}\NormalTok{()}\SpecialCharTok{\}}\SpecialStringTok{"}\NormalTok{)}
\BuiltInTok{print}\NormalTok{(}\SpecialStringTok{f"   Max samples: }\SpecialCharTok{\{}\NormalTok{gas\_counts}\SpecialCharTok{.}\BuiltInTok{max}\NormalTok{()}\SpecialCharTok{\}}\SpecialStringTok{"}\NormalTok{)}
\BuiltInTok{print}\NormalTok{(}\SpecialStringTok{f"   Imbalance ratio: }\SpecialCharTok{\{}\NormalTok{gas\_counts}\SpecialCharTok{.}\BuiltInTok{max}\NormalTok{() }\OperatorTok{/}\NormalTok{ gas\_counts}\SpecialCharTok{.}\BuiltInTok{min}\NormalTok{()}\SpecialCharTok{:.2f\}}\SpecialStringTok{:1"}\NormalTok{)}
\end{Highlighting}
\end{Shaded}

\begin{verbatim}
Original dataset: 13,910 rows × 131 columns

1. Missing Values: 0 total
2. Non-numeric Sensor Columns:  All numeric ✓
3. Duplicate Rows: 0
4. Temporal Consistency:
   Batches present: 1-10
   Missing batches: None ✓
5. Sensor Value Ranges:
   Min: -23660.63
   Max: 670687.35
   Sensors with negative values: 64
6. Sample Balance:
   Gas types: 6 types
   Min samples: 1641
   Max samples: 3009
   Imbalance ratio: 1.83:1
\end{verbatim}

Summary:

\begin{itemize}
\tightlist
\item
  No missing values or duplicates
\item
  All sensor columns numeric
\item
  Complete temporal sequence (batches 1-10)
\item
  Good sample balance (1.83:1 ratio, adequate samples per class)
\end{itemize}

Regarding negative sensor values: These are expected and scientifically
meaningful, as documented by Vergara et al.~(Vergara et al. (2012a)):
``The sensor response to the same gas concentration changed
significantly over time, with some sensors showing sensitivity decrease
(negative drift) and others showing sensitivity increase (positive
drift).'' Therefore, negative readings represent genuine drift phenomena
rather than data errors and should be preserved in our analysis.

\begin{Shaded}
\begin{Highlighting}[]
\CommentTok{\# Group features by type}
\NormalTok{feature\_types }\OperatorTok{=}\NormalTok{ \{}
    \StringTok{\textquotesingle{}DR\textquotesingle{}}\NormalTok{: [col }\ControlFlowTok{for}\NormalTok{ col }\KeywordTok{in}\NormalTok{ sensor\_cols }\ControlFlowTok{if} \StringTok{\textquotesingle{}F1\_DR\textquotesingle{}} \KeywordTok{in}\NormalTok{ col],}
    \StringTok{\textquotesingle{}DR\_norm\textquotesingle{}}\NormalTok{: [col }\ControlFlowTok{for}\NormalTok{ col }\KeywordTok{in}\NormalTok{ sensor\_cols }\ControlFlowTok{if} \StringTok{\textquotesingle{}F2\_DR\_norm\textquotesingle{}} \KeywordTok{in}\NormalTok{ col],}
    \StringTok{\textquotesingle{}EMAi\_001\textquotesingle{}}\NormalTok{: [col }\ControlFlowTok{for}\NormalTok{ col }\KeywordTok{in}\NormalTok{ sensor\_cols }\ControlFlowTok{if} \StringTok{\textquotesingle{}F3\_EMAi\_001\textquotesingle{}} \KeywordTok{in}\NormalTok{ col],}
    \StringTok{\textquotesingle{}EMAi\_01\textquotesingle{}}\NormalTok{: [col }\ControlFlowTok{for}\NormalTok{ col }\KeywordTok{in}\NormalTok{ sensor\_cols }\ControlFlowTok{if} \StringTok{\textquotesingle{}F4\_EMAi\_01\textquotesingle{}} \KeywordTok{in}\NormalTok{ col],}
    \StringTok{\textquotesingle{}EMAi\_1\textquotesingle{}}\NormalTok{: [col }\ControlFlowTok{for}\NormalTok{ col }\KeywordTok{in}\NormalTok{ sensor\_cols }\ControlFlowTok{if} \StringTok{\textquotesingle{}F5\_EMAi\_1\textquotesingle{}} \KeywordTok{in}\NormalTok{ col],}
    \StringTok{\textquotesingle{}EMAd\_001\textquotesingle{}}\NormalTok{: [col }\ControlFlowTok{for}\NormalTok{ col }\KeywordTok{in}\NormalTok{ sensor\_cols }\ControlFlowTok{if} \StringTok{\textquotesingle{}F6\_EMAd\_001\textquotesingle{}} \KeywordTok{in}\NormalTok{ col],}
    \StringTok{\textquotesingle{}EMAd\_01\textquotesingle{}}\NormalTok{: [col }\ControlFlowTok{for}\NormalTok{ col }\KeywordTok{in}\NormalTok{ sensor\_cols }\ControlFlowTok{if} \StringTok{\textquotesingle{}F7\_EMAd\_01\textquotesingle{}} \KeywordTok{in}\NormalTok{ col],}
    \StringTok{\textquotesingle{}EMAd\_1\textquotesingle{}}\NormalTok{: [col }\ControlFlowTok{for}\NormalTok{ col }\KeywordTok{in}\NormalTok{ sensor\_cols }\ControlFlowTok{if} \StringTok{\textquotesingle{}F8\_EMAd\_1\textquotesingle{}} \KeywordTok{in}\NormalTok{ col]}
\NormalTok{\}}

\CommentTok{\# Variance analysis}
\BuiltInTok{print}\NormalTok{(}\StringTok{"FEATURE VARIANCE ANALYSIS"}\NormalTok{)}
\BuiltInTok{print}\NormalTok{(}\StringTok{"="}\OperatorTok{*}\DecValTok{40}\NormalTok{)}
\ControlFlowTok{for}\NormalTok{ i, (ftype, cols) }\KeywordTok{in} \BuiltInTok{enumerate}\NormalTok{(feature\_types.items(), }\DecValTok{1}\NormalTok{):}
    \BuiltInTok{print}\NormalTok{(}\SpecialStringTok{f"F}\SpecialCharTok{\{}\NormalTok{i}\SpecialCharTok{\}}\SpecialStringTok{ (}\SpecialCharTok{\{}\NormalTok{ftype}\SpecialCharTok{:8s\}}\SpecialStringTok{): }\SpecialCharTok{\{}\NormalTok{df[cols]}\SpecialCharTok{.}\NormalTok{var()}\SpecialCharTok{.}\NormalTok{mean()}\SpecialCharTok{:.2e\}}\SpecialStringTok{"}\NormalTok{)}

\CommentTok{\# Sensor health}
\BuiltInTok{print}\NormalTok{(}\SpecialStringTok{f"}\CharTok{\textbackslash{}n}\SpecialStringTok{SENSOR HEALTH: }\SpecialCharTok{\{}\NormalTok{(df[sensor\_cols].var() }\OperatorTok{\textgreater{}} \FloatTok{0.001}\NormalTok{)}\SpecialCharTok{.}\BuiltInTok{sum}\NormalTok{()}\SpecialCharTok{\}}\SpecialStringTok{/}\SpecialCharTok{\{}\BuiltInTok{len}\NormalTok{(sensor\_cols)}\SpecialCharTok{\}}\SpecialStringTok{ healthy"}\NormalTok{)}
\end{Highlighting}
\end{Shaded}

\begin{verbatim}
FEATURE VARIANCE ANALYSIS
========================================
F1 (DR      ): 9.55e+08
F2 (DR_norm ): 5.89e+03
F3 (EMAi_001): 6.90e+01
F4 (EMAi_01 ): 1.88e+02
F5 (EMAi_1  ): 1.24e+03
F6 (EMAd_001): 3.35e+01
F7 (EMAd_01 ): 1.04e+02
F8 (EMAd_1  ): 4.19e+03

SENSOR HEALTH: 128/128 healthy
\end{verbatim}

\begin{Shaded}
\begin{Highlighting}[]
\CommentTok{\# Visualization}
\NormalTok{fig, (ax1, ax2) }\OperatorTok{=}\NormalTok{ plt.subplots(}\DecValTok{1}\NormalTok{, }\DecValTok{2}\NormalTok{, figsize}\OperatorTok{=}\NormalTok{(}\DecValTok{10}\NormalTok{, }\DecValTok{3}\NormalTok{))}

\CommentTok{\# Variance by type}
\NormalTok{means }\OperatorTok{=}\NormalTok{ [df[feature\_types[ft]].var().mean() }\ControlFlowTok{for}\NormalTok{ ft }\KeywordTok{in}\NormalTok{ feature\_types]}
\NormalTok{ax1.bar(}\BuiltInTok{range}\NormalTok{(}\DecValTok{8}\NormalTok{), means, color}\OperatorTok{=}\NormalTok{[}\StringTok{\textquotesingle{}red\textquotesingle{}}\NormalTok{,}\StringTok{\textquotesingle{}orange\textquotesingle{}}\NormalTok{] }\OperatorTok{+}\NormalTok{ [}\StringTok{\textquotesingle{}green\textquotesingle{}}\NormalTok{]}\OperatorTok{*}\DecValTok{3} \OperatorTok{+}\NormalTok{ [}\StringTok{\textquotesingle{}blue\textquotesingle{}}\NormalTok{]}\OperatorTok{*}\DecValTok{3}\NormalTok{, alpha}\OperatorTok{=}\FloatTok{0.7}\NormalTok{)}
\NormalTok{ax1.set\_yscale(}\StringTok{\textquotesingle{}log\textquotesingle{}}\NormalTok{)}
\NormalTok{ax1.set\_xlabel(}\StringTok{\textquotesingle{}Feature Type\textquotesingle{}}\NormalTok{)}
\NormalTok{ax1.set\_ylabel(}\StringTok{\textquotesingle{}Mean Variance (log)\textquotesingle{}}\NormalTok{)}
\NormalTok{ax1.set\_title(}\StringTok{\textquotesingle{}Variance by Feature Type\textquotesingle{}}\NormalTok{)}
\NormalTok{ax1.set\_xticklabels([}\SpecialStringTok{f\textquotesingle{}F}\SpecialCharTok{\{}\NormalTok{i}\OperatorTok{+}\DecValTok{1}\SpecialCharTok{\}}\SpecialStringTok{\textquotesingle{}} \ControlFlowTok{for}\NormalTok{ i }\KeywordTok{in} \BuiltInTok{range}\NormalTok{(}\DecValTok{8}\NormalTok{)])}

\CommentTok{\# Heatmap}
\NormalTok{variance\_matrix }\OperatorTok{=}\NormalTok{ np.zeros((}\DecValTok{16}\NormalTok{, }\DecValTok{8}\NormalTok{))}
\ControlFlowTok{for}\NormalTok{ s }\KeywordTok{in} \BuiltInTok{range}\NormalTok{(}\DecValTok{16}\NormalTok{):}
    \ControlFlowTok{for}\NormalTok{ f }\KeywordTok{in} \BuiltInTok{range}\NormalTok{(}\DecValTok{8}\NormalTok{):}
\NormalTok{        col }\OperatorTok{=} \SpecialStringTok{f\textquotesingle{}S}\SpecialCharTok{\{}\NormalTok{s}\OperatorTok{+}\DecValTok{1}\SpecialCharTok{:02d\}}\SpecialStringTok{\_F}\SpecialCharTok{\{}\NormalTok{f}\OperatorTok{+}\DecValTok{1}\SpecialCharTok{\}}\SpecialStringTok{\_\textquotesingle{}} \OperatorTok{+} \BuiltInTok{list}\NormalTok{(feature\_types.keys())[f].split(}\StringTok{\textquotesingle{}\_\textquotesingle{}}\NormalTok{)[}\DecValTok{0}\NormalTok{]}
\NormalTok{        matching }\OperatorTok{=}\NormalTok{ [c }\ControlFlowTok{for}\NormalTok{ c }\KeywordTok{in}\NormalTok{ sensor\_cols }\ControlFlowTok{if}\NormalTok{ c.startswith(}\SpecialStringTok{f\textquotesingle{}S}\SpecialCharTok{\{}\NormalTok{s}\OperatorTok{+}\DecValTok{1}\SpecialCharTok{:02d\}}\SpecialStringTok{\_F}\SpecialCharTok{\{}\NormalTok{f}\OperatorTok{+}\DecValTok{1}\SpecialCharTok{\}}\SpecialStringTok{\_\textquotesingle{}}\NormalTok{)]}
        \ControlFlowTok{if}\NormalTok{ matching:}
\NormalTok{            variance\_matrix[s, f] }\OperatorTok{=}\NormalTok{ df[matching[}\DecValTok{0}\NormalTok{]].var()}

\NormalTok{im }\OperatorTok{=}\NormalTok{ ax2.imshow(variance\_matrix, aspect}\OperatorTok{=}\StringTok{\textquotesingle{}auto\textquotesingle{}}\NormalTok{, cmap}\OperatorTok{=}\StringTok{\textquotesingle{}YlOrRd\textquotesingle{}}\NormalTok{, norm}\OperatorTok{=}\NormalTok{plt.matplotlib.colors.LogNorm())}
\NormalTok{ax2.set\_xlabel(}\StringTok{\textquotesingle{}Feature\textquotesingle{}}\NormalTok{)}
\NormalTok{ax2.set\_ylabel(}\StringTok{\textquotesingle{}Sensor\textquotesingle{}}\NormalTok{)}
\NormalTok{ax2.set\_title(}\StringTok{\textquotesingle{}Variance Heatmap\textquotesingle{}}\NormalTok{)}
\NormalTok{plt.colorbar(im, ax}\OperatorTok{=}\NormalTok{ax2, fraction}\OperatorTok{=}\FloatTok{0.046}\NormalTok{, pad}\OperatorTok{=}\FloatTok{0.04}\NormalTok{)}
\NormalTok{plt.tight\_layout()}
\NormalTok{plt.show()}
\end{Highlighting}
\end{Shaded}

\pandocbounded{\includegraphics[keepaspectratio]{main_files/figure-pdf/cell-5-output-1.png}}

\subsection{Data Quality Assessment}\label{data-quality-assessment}

Data quality assessment revealed no missing values, duplicates, or
measurement errors requiring correction. The 64 sensors with negative
values represent genuine drift phenomena (sensitivity decreases)
documented by Vergara et al.~(2012), not errors.

\section{Exploratory Data Analysis}\label{exploratory-data-analysis}

\subsection{Target Variable
Distribution}\label{target-variable-distribution}

\begin{Shaded}
\begin{Highlighting}[]
\CommentTok{\# Example: Analyze target distribution}
\CommentTok{\# df[\textquotesingle{}target\textquotesingle{}].value\_counts()}
\end{Highlighting}
\end{Shaded}

\subsection{Feature Analysis}\label{feature-analysis}

\begin{Shaded}
\begin{Highlighting}[]
\CommentTok{\# Example visualization}
\ImportTok{import}\NormalTok{ matplotlib.pyplot }\ImportTok{as}\NormalTok{ plt}

\CommentTok{\# Add your visualizations here}
\end{Highlighting}
\end{Shaded}

\section{Models}\label{models}

\subsection{Model Selection}\label{model-selection}

{[}Describe chosen models and rationale{]}

\subsection{Hyperparameter
Configuration}\label{hyperparameter-configuration}

{[}Document hyperparameter choices{]}

\section{Training}\label{training}

\subsection{Training Pipeline}\label{training-pipeline}

\begin{Shaded}
\begin{Highlighting}[]
\CommentTok{\# Example: Model training}
\ImportTok{from}\NormalTok{ sklearn.model\_selection }\ImportTok{import}\NormalTok{ train\_test\_split}

\CommentTok{\# Split data}
\CommentTok{\# X\_train, X\_test, y\_train, y\_test = train\_test\_split(...)}
\end{Highlighting}
\end{Shaded}

\subsection{Training Results}\label{training-results}

{[}Present initial training metrics{]}

\section{Evaluation}\label{evaluation}

\subsection{Performance Metrics}\label{performance-metrics}

\begin{Shaded}
\begin{Highlighting}[]
\CommentTok{\# Example: Evaluation metrics}
\CommentTok{\# from sklearn.metrics import classification\_report}
\end{Highlighting}
\end{Shaded}

\subsection{Model Comparison}\label{model-comparison}

{[}Compare different models' performance{]}

\section{Conclusion and Next Steps}\label{conclusion-and-next-steps}

\subsection{Project Summary}\label{project-summary}

{[}Summarize key findings{]}

\subsection{Recommendations}\label{recommendations}

{[}Provide actionable recommendations{]}

\subsection{Future Work}\label{future-work}

{[}Outline potential improvements and extensions{]}

\section{References}\label{references}

{[}List relevant citations and data sources{]}

\phantomsection\label{refs}
\begin{CSLReferences}{1}{0}
\bibitem[\citeproctext]{ref-rodriguez2014calibration}
Rodriguez-Lujan, Irene, Jordi Fonollosa, Alexander Vergara, Margie
Homer, and Ramón Huerta. 2014. {``On the Calibration of Sensor Arrays
for Pattern Recognition Using the Minimal Number of Experiments.''} In
\emph{Chemometrics and Intelligent Laboratory Systems}, 130:123--34.
Elsevier. \url{https://doi.org/10.1016/j.chemolab.2013.10.012}.

\bibitem[\citeproctext]{ref-vergara2012chemical}
Vergara, Alexander, Shankar Vembu, Tuba Ayhan, Margaret A. Ryan, Margie
L. Homer, and Ramón Huerta. 2012a. {``Chemical Gas Sensor Drift
Compensation Using Classifier Ensembles.''} \emph{Sensors and Actuators
B: Chemical} 166: 320--29.
\url{https://doi.org/10.1016/j.snb.2012.01.074}.

\bibitem[\citeproctext]{ref-vergara2012gas}
---------. 2012b. {``Gas Sensor Array Drift Dataset.''}
\url{https://archive.ics.uci.edu/ml/datasets/Gas+Sensor+Array+Drift+Dataset};
UCI Machine Learning Repository. \url{https://doi.org/10.24432/C5QC8P}.

\end{CSLReferences}




\end{document}
