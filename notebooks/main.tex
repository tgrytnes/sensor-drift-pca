% Options for packages loaded elsewhere
% Options for packages loaded elsewhere
\PassOptionsToPackage{unicode}{hyperref}
\PassOptionsToPackage{hyphens}{url}
\PassOptionsToPackage{dvipsnames,svgnames,x11names}{xcolor}
%
\documentclass[
  letterpaper,
  DIV=11,
  numbers=noendperiod]{scrartcl}
\usepackage{xcolor}
\usepackage[margin=1in]{geometry}
\usepackage{amsmath,amssymb}
\setcounter{secnumdepth}{5}
\usepackage{iftex}
\ifPDFTeX
  \usepackage[T1]{fontenc}
  \usepackage[utf8]{inputenc}
  \usepackage{textcomp} % provide euro and other symbols
\else % if luatex or xetex
  \usepackage{unicode-math} % this also loads fontspec
  \defaultfontfeatures{Scale=MatchLowercase}
  \defaultfontfeatures[\rmfamily]{Ligatures=TeX,Scale=1}
\fi
\usepackage{lmodern}
\ifPDFTeX\else
  % xetex/luatex font selection
  \setmainfont[]{Latin Modern Roman}
  \setsansfont[]{Latin Modern Sans}
  \setmonofont[]{Latin Modern Mono}
\fi
% Use upquote if available, for straight quotes in verbatim environments
\IfFileExists{upquote.sty}{\usepackage{upquote}}{}
\IfFileExists{microtype.sty}{% use microtype if available
  \usepackage[]{microtype}
  \UseMicrotypeSet[protrusion]{basicmath} % disable protrusion for tt fonts
}{}
\makeatletter
\@ifundefined{KOMAClassName}{% if non-KOMA class
  \IfFileExists{parskip.sty}{%
    \usepackage{parskip}
  }{% else
    \setlength{\parindent}{0pt}
    \setlength{\parskip}{6pt plus 2pt minus 1pt}}
}{% if KOMA class
  \KOMAoptions{parskip=half}}
\makeatother
% Make \paragraph and \subparagraph free-standing
\makeatletter
\ifx\paragraph\undefined\else
  \let\oldparagraph\paragraph
  \renewcommand{\paragraph}{
    \@ifstar
      \xxxParagraphStar
      \xxxParagraphNoStar
  }
  \newcommand{\xxxParagraphStar}[1]{\oldparagraph*{#1}\mbox{}}
  \newcommand{\xxxParagraphNoStar}[1]{\oldparagraph{#1}\mbox{}}
\fi
\ifx\subparagraph\undefined\else
  \let\oldsubparagraph\subparagraph
  \renewcommand{\subparagraph}{
    \@ifstar
      \xxxSubParagraphStar
      \xxxSubParagraphNoStar
  }
  \newcommand{\xxxSubParagraphStar}[1]{\oldsubparagraph*{#1}\mbox{}}
  \newcommand{\xxxSubParagraphNoStar}[1]{\oldsubparagraph{#1}\mbox{}}
\fi
\makeatother

\usepackage{color}
\usepackage{fancyvrb}
\newcommand{\VerbBar}{|}
\newcommand{\VERB}{\Verb[commandchars=\\\{\}]}
\DefineVerbatimEnvironment{Highlighting}{Verbatim}{commandchars=\\\{\}}
% Add ',fontsize=\small' for more characters per line
\usepackage{framed}
\definecolor{shadecolor}{RGB}{241,243,245}
\newenvironment{Shaded}{\begin{snugshade}}{\end{snugshade}}
\newcommand{\AlertTok}[1]{\textcolor[rgb]{0.68,0.00,0.00}{#1}}
\newcommand{\AnnotationTok}[1]{\textcolor[rgb]{0.37,0.37,0.37}{#1}}
\newcommand{\AttributeTok}[1]{\textcolor[rgb]{0.40,0.45,0.13}{#1}}
\newcommand{\BaseNTok}[1]{\textcolor[rgb]{0.68,0.00,0.00}{#1}}
\newcommand{\BuiltInTok}[1]{\textcolor[rgb]{0.00,0.23,0.31}{#1}}
\newcommand{\CharTok}[1]{\textcolor[rgb]{0.13,0.47,0.30}{#1}}
\newcommand{\CommentTok}[1]{\textcolor[rgb]{0.37,0.37,0.37}{#1}}
\newcommand{\CommentVarTok}[1]{\textcolor[rgb]{0.37,0.37,0.37}{\textit{#1}}}
\newcommand{\ConstantTok}[1]{\textcolor[rgb]{0.56,0.35,0.01}{#1}}
\newcommand{\ControlFlowTok}[1]{\textcolor[rgb]{0.00,0.23,0.31}{\textbf{#1}}}
\newcommand{\DataTypeTok}[1]{\textcolor[rgb]{0.68,0.00,0.00}{#1}}
\newcommand{\DecValTok}[1]{\textcolor[rgb]{0.68,0.00,0.00}{#1}}
\newcommand{\DocumentationTok}[1]{\textcolor[rgb]{0.37,0.37,0.37}{\textit{#1}}}
\newcommand{\ErrorTok}[1]{\textcolor[rgb]{0.68,0.00,0.00}{#1}}
\newcommand{\ExtensionTok}[1]{\textcolor[rgb]{0.00,0.23,0.31}{#1}}
\newcommand{\FloatTok}[1]{\textcolor[rgb]{0.68,0.00,0.00}{#1}}
\newcommand{\FunctionTok}[1]{\textcolor[rgb]{0.28,0.35,0.67}{#1}}
\newcommand{\ImportTok}[1]{\textcolor[rgb]{0.00,0.46,0.62}{#1}}
\newcommand{\InformationTok}[1]{\textcolor[rgb]{0.37,0.37,0.37}{#1}}
\newcommand{\KeywordTok}[1]{\textcolor[rgb]{0.00,0.23,0.31}{\textbf{#1}}}
\newcommand{\NormalTok}[1]{\textcolor[rgb]{0.00,0.23,0.31}{#1}}
\newcommand{\OperatorTok}[1]{\textcolor[rgb]{0.37,0.37,0.37}{#1}}
\newcommand{\OtherTok}[1]{\textcolor[rgb]{0.00,0.23,0.31}{#1}}
\newcommand{\PreprocessorTok}[1]{\textcolor[rgb]{0.68,0.00,0.00}{#1}}
\newcommand{\RegionMarkerTok}[1]{\textcolor[rgb]{0.00,0.23,0.31}{#1}}
\newcommand{\SpecialCharTok}[1]{\textcolor[rgb]{0.37,0.37,0.37}{#1}}
\newcommand{\SpecialStringTok}[1]{\textcolor[rgb]{0.13,0.47,0.30}{#1}}
\newcommand{\StringTok}[1]{\textcolor[rgb]{0.13,0.47,0.30}{#1}}
\newcommand{\VariableTok}[1]{\textcolor[rgb]{0.07,0.07,0.07}{#1}}
\newcommand{\VerbatimStringTok}[1]{\textcolor[rgb]{0.13,0.47,0.30}{#1}}
\newcommand{\WarningTok}[1]{\textcolor[rgb]{0.37,0.37,0.37}{\textit{#1}}}

\usepackage{longtable,booktabs,array}
\usepackage{calc} % for calculating minipage widths
% Correct order of tables after \paragraph or \subparagraph
\usepackage{etoolbox}
\makeatletter
\patchcmd\longtable{\par}{\if@noskipsec\mbox{}\fi\par}{}{}
\makeatother
% Allow footnotes in longtable head/foot
\IfFileExists{footnotehyper.sty}{\usepackage{footnotehyper}}{\usepackage{footnote}}
\makesavenoteenv{longtable}
\usepackage{graphicx}
\makeatletter
\newsavebox\pandoc@box
\newcommand*\pandocbounded[1]{% scales image to fit in text height/width
  \sbox\pandoc@box{#1}%
  \Gscale@div\@tempa{\textheight}{\dimexpr\ht\pandoc@box+\dp\pandoc@box\relax}%
  \Gscale@div\@tempb{\linewidth}{\wd\pandoc@box}%
  \ifdim\@tempb\p@<\@tempa\p@\let\@tempa\@tempb\fi% select the smaller of both
  \ifdim\@tempa\p@<\p@\scalebox{\@tempa}{\usebox\pandoc@box}%
  \else\usebox{\pandoc@box}%
  \fi%
}
% Set default figure placement to htbp
\def\fps@figure{htbp}
\makeatother





\setlength{\emergencystretch}{3em} % prevent overfull lines

\providecommand{\tightlist}{%
  \setlength{\itemsep}{0pt}\setlength{\parskip}{0pt}}



 


% Colors and section/title styling using KOMA-Script interfaces
\usepackage{xcolor}
\definecolor{sectionblue}{HTML}{2563eb}

% KOMA: headings and title/subtitle colors
\setkomafont{title}{\color{sectionblue}\bfseries\Huge}
\setkomafont{subtitle}{\color{sectionblue}\large}
\setkomafont{section}{\color{sectionblue}\bfseries\Large}
\setkomafont{subsection}{\color{sectionblue}\bfseries\large}

% Code block styling via Shaded redefinition
\usepackage{tcolorbox}
\tcbuselibrary{skins,breakable}
\definecolor{codebg}{HTML}{F0F8FF}
\renewenvironment{Shaded}{%
  \begin{tcolorbox}[%
    enhanced,%
    colback=codebg,%
    colframe=codebg,%
    borderline west={3pt}{0pt}{sectionblue},%
    fontupper=\small\ttfamily,% reduce font size and force monospace for code
    boxrule=0pt,%
    arc=0pt,%
    boxsep=5pt,%
    left=2mm,%
    right=2mm,%
    top=2mm,%
    bottom=2mm% 
  ]% 
}{%
  \end{tcolorbox}%
}
\KOMAoption{captions}{tableheading}
\makeatletter
\@ifpackageloaded{caption}{}{\usepackage{caption}}
\AtBeginDocument{%
\ifdefined\contentsname
  \renewcommand*\contentsname{Table of contents}
\else
  \newcommand\contentsname{Table of contents}
\fi
\ifdefined\listfigurename
  \renewcommand*\listfigurename{List of Figures}
\else
  \newcommand\listfigurename{List of Figures}
\fi
\ifdefined\listtablename
  \renewcommand*\listtablename{List of Tables}
\else
  \newcommand\listtablename{List of Tables}
\fi
\ifdefined\figurename
  \renewcommand*\figurename{Figure}
\else
  \newcommand\figurename{Figure}
\fi
\ifdefined\tablename
  \renewcommand*\tablename{Table}
\else
  \newcommand\tablename{Table}
\fi
}
\@ifpackageloaded{float}{}{\usepackage{float}}
\floatstyle{ruled}
\@ifundefined{c@chapter}{\newfloat{codelisting}{h}{lop}}{\newfloat{codelisting}{h}{lop}[chapter]}
\floatname{codelisting}{Listing}
\newcommand*\listoflistings{\listof{codelisting}{List of Listings}}
\makeatother
\makeatletter
\makeatother
\makeatletter
\@ifpackageloaded{caption}{}{\usepackage{caption}}
\@ifpackageloaded{subcaption}{}{\usepackage{subcaption}}
\makeatother
\usepackage{bookmark}
\IfFileExists{xurl.sty}{\usepackage{xurl}}{} % add URL line breaks if available
\urlstyle{same}
\hypersetup{
  pdftitle={Gas Sensor Drift Analysis with PCA},
  colorlinks=true,
  linkcolor={blue},
  filecolor={Maroon},
  citecolor={Blue},
  urlcolor={Blue},
  pdfcreator={LaTeX via pandoc}}


\title{Gas Sensor Drift Analysis with PCA}
\usepackage{etoolbox}
\makeatletter
\providecommand{\subtitle}[1]{% add subtitle to \maketitle
  \apptocmd{\@title}{\par {\large #1 \par}}{}{}
}
\makeatother
\subtitle{Leak-Aware Prediction at Booking Time}
\author{}
\date{}
\begin{document}
\maketitle

\renewcommand*\contentsname{Table of contents}
{
\hypersetup{linkcolor=}
\setcounter{tocdepth}{3}
\tableofcontents
}

\section{Project Introduction and Problem
Description}\label{project-introduction-and-problem-description}

\subsection{Project Introduction}\label{project-introduction}

{[}Brief description of the project goals and business context{]}

\subsection{Learning Approach}\label{learning-approach}

{[}Describe the machine learning approach - supervised/unsupervised,
classification/regression, etc.{]}

\subsection{Model Strategy}\label{model-strategy}

{[}Outline the modeling approaches to be used and why{]}

\subsection{Evaluation Framework}\label{evaluation-framework}

{[}Define success metrics and validation strategy{]}

\section{Data Loading \& Initial
Inspection}\label{data-loading-initial-inspection}

\subsection{Dataset Description}\label{dataset-description}

{[}Describe the dataset source, size, and key characteristics{]}

\begin{Shaded}
\begin{Highlighting}[]
\ImportTok{import}\NormalTok{ pandas }\ImportTok{as}\NormalTok{ pd}
\ImportTok{import}\NormalTok{ numpy }\ImportTok{as}\NormalTok{ np}
\ImportTok{from}\NormalTok{ pathlib }\ImportTok{import}\NormalTok{ Path}

\CommentTok{\# Load data}
\NormalTok{data\_path }\OperatorTok{=}\NormalTok{ Path(}\StringTok{"../data/processed/sensor\_data.csv"}\NormalTok{)}
\NormalTok{df }\OperatorTok{=}\NormalTok{ pd.read\_csv(data\_path)}

\CommentTok{\# Basic information}
\BuiltInTok{print}\NormalTok{(}\SpecialStringTok{f"Dataset shape: }\SpecialCharTok{\{}\NormalTok{df}\SpecialCharTok{.}\NormalTok{shape[}\DecValTok{0}\NormalTok{]}\SpecialCharTok{:,\}}\SpecialStringTok{ rows × }\SpecialCharTok{\{}\NormalTok{df}\SpecialCharTok{.}\NormalTok{shape[}\DecValTok{1}\NormalTok{]}\SpecialCharTok{\}}\SpecialStringTok{ columns"}\NormalTok{)}
\BuiltInTok{print}\NormalTok{(}\SpecialStringTok{f"}\CharTok{\textbackslash{}n}\SpecialStringTok{Data types: }\SpecialCharTok{\{}\NormalTok{df}\SpecialCharTok{.}\NormalTok{dtypes}\SpecialCharTok{.}\NormalTok{value\_counts()}\SpecialCharTok{.}\NormalTok{to\_dict()}\SpecialCharTok{\}}\SpecialStringTok{"}\NormalTok{)}
\end{Highlighting}
\end{Shaded}

\section{Preprocess and Data
Cleaning}\label{preprocess-and-data-cleaning}

\subsection{Missing Value Treatment}\label{missing-value-treatment}

{[}Document data cleaning steps{]}

\begin{Shaded}
\begin{Highlighting}[]
\CommentTok{\# Example: Handle missing values}
\NormalTok{df\_cleaned }\OperatorTok{=}\NormalTok{ df.copy()}
\CommentTok{\# Add cleaning steps here}
\end{Highlighting}
\end{Shaded}

\section{Exploratory Data Analysis}\label{exploratory-data-analysis}

\subsection{Target Variable
Distribution}\label{target-variable-distribution}

\begin{Shaded}
\begin{Highlighting}[]
\CommentTok{\# Example: Analyze target distribution}
\CommentTok{\# df[\textquotesingle{}target\textquotesingle{}].value\_counts()}
\end{Highlighting}
\end{Shaded}

\subsection{Feature Analysis}\label{feature-analysis}

\begin{Shaded}
\begin{Highlighting}[]
\CommentTok{\# Example visualization}
\ImportTok{import}\NormalTok{ matplotlib.pyplot }\ImportTok{as}\NormalTok{ plt}

\CommentTok{\# Add your visualizations here}
\end{Highlighting}
\end{Shaded}

\section{Models}\label{models}

\subsection{Model Selection}\label{model-selection}

{[}Describe chosen models and rationale{]}

\subsection{Hyperparameter
Configuration}\label{hyperparameter-configuration}

{[}Document hyperparameter choices{]}

\section{Training}\label{training}

\subsection{Training Pipeline}\label{training-pipeline}

\begin{Shaded}
\begin{Highlighting}[]
\CommentTok{\# Example: Model training}
\ImportTok{from}\NormalTok{ sklearn.model\_selection }\ImportTok{import}\NormalTok{ train\_test\_split}

\CommentTok{\# Split data}
\CommentTok{\# X\_train, X\_test, y\_train, y\_test = train\_test\_split(...)}
\end{Highlighting}
\end{Shaded}

\subsection{Training Results}\label{training-results}

{[}Present initial training metrics{]}

\section{Evaluation}\label{evaluation}

\subsection{Performance Metrics}\label{performance-metrics}

\begin{Shaded}
\begin{Highlighting}[]
\CommentTok{\# Example: Evaluation metrics}
\CommentTok{\# from sklearn.metrics import classification\_report}
\end{Highlighting}
\end{Shaded}

\subsection{Model Comparison}\label{model-comparison}

{[}Compare different models' performance{]}

\section{Conclusion and Next Steps}\label{conclusion-and-next-steps}

\subsection{Project Summary}\label{project-summary}

{[}Summarize key findings{]}

\subsection{Recommendations}\label{recommendations}

{[}Provide actionable recommendations{]}

\subsection{Future Work}\label{future-work}

{[}Outline potential improvements and extensions{]}

\section{References}\label{references}

{[}List relevant citations and data sources{]}




\end{document}
